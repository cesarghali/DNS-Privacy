% THIS IS SIGPROC-SP.TEX - VERSION 3.1
% WORKS WITH V3.2SP OF ACM_PROC_ARTICLE-SP.CLS
% APRIL 2009

\documentclass{llncs}

\usepackage{todonotes,times}
\usepackage{algorithm}
\usepackage{url}
\usepackage[noend]{algpseudocode}
\usepackage{mdframed}
\usepackage{amsmath}
\usepackage{paralist}
\usepackage{multirow,paralist}
\usepackage{fancybox}

\newenvironment{fminipage}%
{\begin{Sbox}\begin{minipage}}%
{\end{minipage}\end{Sbox}\fbox{\TheSbox}}

\makeatletter
%\renewcommand{\ALG@beginalgorithmic}{\scriptsize}
\makeatother

\newtheorem{defn}{\textbf{Definition}}
\newtheorem{thm}{\textbf{Theorem}}
\newtheorem{cor}{\textbf{Corollary}}
%\newtheorem{lemma}{\textbf{Lemma}}

\begin{document}

\mainmatter              % start of the contributions
\title{On the Efficacy of DNS Resolver Privacy Preservation}
\titlerunning{TODO}

\author{Cesar Ghali, Gene Tsudik, Christopher A. Wood}
\authorrunning{Wood} % abbreviated author list (for running head)
%%%% list of authors for the TOC (use if author list has to be modified)
\tocauthor{Cesar Ghali, Gene Tsudik, Christopher A. Wood}

\institute{University of California Irvine, Irvine CA, USA\\
\email{\{cghali,gene.tsudik,woodc1@\}uci.edu}}

% typeset the title of the contribution
\maketitle

\begin{abstract}
TODO
\end{abstract}

\section{Introduction}
The need for a private Domain Name System (DNS) has become increasingly important
in recent years. There are currently several different proposals to address this
growing problem, including DNS-over-TLS \cite{dnstls} and DNSCurve \cite{dnscurve}.
The former approach enables clients to create ephemeral sessions with either
their resolver or authoritative (stub) servers in which queries can be issued.
The latter uses per-query encryption to protect queries between clients and servers.
Encryption is core mechanism used to enable client privacy in both of these 
solutions. However, in a recent study, Shulman showed that the privacy properties
of these solutions (based on encryption alone) against eavesdropping 
adversaries \cite{shulman2014pretty}. This assessment showed that information leaked
in DNS side channels, e.g., query timing, frequency, and resolution ``chains,'' 
may reveal the target domain for a given DNS query. Moreover, by observing the
trust properties of DNS servers and their responses, an adversary may also 
learn the specific record within a domain that was requested. 

In this work we study a complementary problem. Namely, how can DNS queries, 
encrypted or not, be used to identify their clients? Put another way, do the 
contents of queries recursively issued by resolvers reveal information about
the resolver's clients? This is an important problem because if answered
positively, then stub servers can learn information about DNS clients even
if encryption is used to protect the actual contents of queries in transit. 

The rest of this report is organized as follows. In Section \ref{sec:model}
we formalize the adversarial model and 

\section{System Model}\label{sec:model}
To assess the 


If an adversary cannot learn any information from the name of an interest (and content object)
beyond the destination, then it must be true that
\begin{align*}
\Pr[\mathsf{AddressPrefix}_{\mathcal{A}}(1^\lambda) = 1]\leq \frac{1}{2} + \epsilon(\lambda)
\end{align*}

\section{Overview}
TODO

\begin{algorithm}[t]
  \caption{Random Onion Routing}
  \begin{algorithmic}[1]
    % \Require{INPUT}

\Function{{\sf WrapName}}{$n = [n_1,\dots,n_l]$, $k = [k_1,\dots,k_l]$}
    \State $\bar{n} := []$
    \For{$i = 1 \to l$}
    	\State $\mathsf{prefix} := H(n[1:i]) \oplus k_i$
    	\State $\mathsf{Append}(\bar{n}, \mathsf{prefix})$
    \EndFor
    \State \textbf{return} $\bar{n}$
\EndFunction

\Function{{\sf EncodeName}}{$n = [n_1,\dots,n_l]$, $k = [k_1,\dots,k_l]$, $r = [r_1,\dots,r_l]$}
    \State $\bar{n} := []$
    \For{$i = 1 \to l$}
    	\State $\mathsf{prefix} := n[1:i] \oplus k_i \oplus r_i$
    	\State $\mathsf{Append}(\bar{n}, \mathsf{prefix})$
    \EndFor
    \State \textbf{return} $\bar{n}$
\EndFunction

\Function{{\sf PropogateName}}{$R_{up}$, $IF_{up}$, $n$, $k_R = [k_1,\dots,k_l]$, $r_R = [r_1,\dots,r_l]$, $s_R = [s_1,\dots,s_l]$} % s is the secret mask used for wrapping
	\State $\mathbf{R}_{out} = \mathsf{GetOutputInterfaces(P, R_{up})}$ % all output interfaces
	\State $\hat{n} := \mathsf{EncodeName}(n, k_R, r_R)$
	\State Insert each prefix of $\hat{n}$ into the local FIB with the tuple $(n, IF_{up})$
	\State $\bar{n} := \mathsf{WrapName}(n, s_R)$
    \For{$i = 1 \to |\mathbf{R}|$}
      \State $\mathsf{PropogateName}(\mathbf{R}[i], \bar{n})$
    \EndFor
\EndFunction

\Function{{\sf Advertise}}{$P$, $n$, $s_P = [s_1,\dots,s_l]$}
    \State $\bar{n} := \mathsf{WrapName}(n, s_P)$
    \State $\mathbf{R}_{out} = \mathsf{GetOutputInterfaces(P, \emptyset)}$
    \For{$i = 1 \to |\mathbf{R}|$}
    	\State $\mathsf{PropogateName}(\mathbf{R}[i], \bar{n})$
    \EndFor
\EndFunction

\end{algorithmic}
\end{algorithm}

\begin{algorithm}[t]
  \caption{Random Onion Forwarding}
  \begin{algorithmic}[1]
    % \Require{INPUT}

\Function{{\sf DecodeName}}{$n = [n_1,\dots,n_l]$, $k = [k_1,\dots,k_l]$, $r = [r_1,\dots,r_l]$}
    \State $\bar{n} := []$
    \For{$i = 1 \to l$}
    	\State $\mathsf{prefix} := n[1:i] \oplus k_i \oplus r_i$
    	\State $\mathsf{Append}(\bar{n}, \mathsf{prefix})$
    \EndFor
    \State \textbf{return} $\bar{n}$
\EndFunction

\Function{{\sf UnwrapName}}{$n = [n_1,\dots,n_l]$, $r = [r_1,\dots,r_l]$}
    \State $\bar{n} := []$
    \For{$i = 1 \to l$}
    	\State $\mathsf{prefix} := n[1:i] \oplus r_i$
    	\State $\mathsf{Append}(\bar{n}, \mathsf{prefix})$
    \EndFor
    \State \textbf{return} $\bar{n}$
\EndFunction

\Function{{\sf ForwarderAccept}}{$n = [n_1,\dots,n_l]$, $k_R = [k_1,\dots,k_l]$, $r = [r_1,\dots,r_l]$, $s_R = [s_1,\dots,s_l]$}
	\State $\bar{n} := \mathsf{UnwrapName}(n, s_R)$ %%% we can remove its mask, but then n will still be masked by all other upstream strings (S_2,\dots,S_l)
	\If {$\bar{n}$ in FIB}
		\State $(n, IF_{up}) := \mathsf{FIB}.\mathsf{Lookup}(\bar{n})$
		\State $\mathsf{ForwarderNameProcess}(IF_{up}, n, k_R)$
	\Else
		\State Drop interest
	\EndIf
\EndFunction

\Function{{\sf ForwarderNameProcess}}{$IF_{up}$, $n = [n_1,\dots,n_l]$, $k_R = [k_1,\dots,k_l]$}
    \State $k_{up}, r_{up} := \mathsf{GetUpstreamParams}(R_{up})$
    \State $\bar{n} := \mathsf{EncodeName}(n, k_{up}, r_{up})$
    \State $\hat{r} := []$ % initialize r with random masks for each of the l components
    \For {$i := 1 \to l$}
    	\State $rand \gets \{0,1\}^{\lambda}$
    	\State $\mathsf{Append}(\hat{r}, rand)$
    \EndFor
    \State $\hat{n} := \mathsf{DecodeName}(\bar{n}, k_{up}, \hat{r})$ % new plaintext name, to use with random IV
    \State Send $(\hat{n}, \hat{r})$ to $IF_{up}$
\EndFunction

\end{algorithmic}
\end{algorithm}

\medskip
\small
\bibliographystyle{abbrv}
\bibliography{ref}

\end{document}
